\documentclass[11pt]{article}   

\usepackage{geometry}                		% See geometry.pdf to learn the
\usepackage{fixltx2e}
\geometry{letterpaper}                   		% ... or a4paper or
\usepackage{amsmath}
\usepackage{amssymb}
\usepackage{color}
\usepackage{subfigure}
\usepackage[font={bf,small},labelsep=colon]{caption}

% \usepackage{times} % Use Times Font (appear line fonts in MS Word)
%\usepackage{mathptmx}

%\usepackage[parfill]{parskip}    		
% Activate to begin paragraphs with an empty line rather than an indent 

\usepackage{graphicx}				% Use pdf, png, jpg, or eps§
\usepackage{caption}
\usepackage{hyperref}

\geometry{
  body={6.5in, 9.0in},
  left=1.0in,
  top=1.0in
}


\renewcommand{\baselinestretch}{1}

\title{Eulerian--Lagrangian Modeling of Current-Induced Coastal Sand Dune Migration}

\author{Rui Sun \and Heng Xiao}

\date{\today}
\begin{document}

\maketitle

\section{Validation cases}

Several numerical example cases are presented in this section as the validation of lammpsFoam. In
Case 1, a sand bed is fluidized by airflow in the upward direction with a gradually increasing and
decreasing velocity, aiming at capturing the fluidized velocity and pressure drop. In Case 2, a
similar simulation is performed with Case 1, but the fluidized bed is made of Perspex and the
airflow velocity is constant. The volume fraction of the fluid and the time-averaged Eulerian
velocity of particles are presented in this case.

\subsection{Case 1: fluidized bed with constant air flow velocity}

\subsection{Case 2: fluidized bed with oscillating air flow velocity}

\subsection{Code validation part}

In this part, the results of different versions of code are compared to see if the modification of
the code makes any sense.

