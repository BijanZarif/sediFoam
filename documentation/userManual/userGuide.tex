% \documentclass[11pt, twocolumn]{article}
\documentclass[11pt]{article}   
\usepackage{titlesec}
\usepackage{algorithm2e}
\usepackage[usenames]{color}
\usepackage{lineno}

\setcounter{secnumdepth}{4}

\titleformat{\paragraph}
{\normalfont\normalsize\bfseries}{\theparagraph}{1em}{}
\titlespacing*{\paragraph}
{0pt}{3.25ex plus 1ex minus .2ex}{1.5ex plus .2ex}

\usepackage{inputenc}
\usepackage{tikz}
\usetikzlibrary{shapes,arrows}
\usepackage{geometry}                		% See geometry.pdf to learn the
\usepackage{fixltx2e}
\geometry{letterpaper}                   		% ... or a4paper or
\usepackage{amsmath}
\usepackage{amssymb}
\usepackage{sectsty}
\usepackage{wrapfig}
\usepackage{color}
\usepackage{subfigure}
\usepackage[font={bf,small},labelsep=colon]{caption}
\usepackage{authblk}
\usepackage{booktabs}               % For drawing a table
\usepackage{listings}               % For drawing a table

%\usepackage[sort&compress, numbers]{natbib}

% \usepackage{times} % Use Times Font (appear like fonts in MS Word)
%\usepackage{mathptmx}

%\usepackage[parfill]{parskip}    		
% Activate to begin paragraphs with an empty line rather than an indent 

\usepackage{graphicx}				% Use pdf, png, jpg, or eps§
\usepackage{caption}
%\usepackage{subcaption}            % Use of subcaption, do not understand
\usepackage{hyperref}


% \usepackage[none]{hyphenat} % prevent hypenating; not a good idea

\newcommand*{\TitleFont}{%
       \usefont{\encodingdefault}{\rmdefault}{b}{n}%
       \fontsize{14}{14}%
       \selectfont}
\newcommand\abs[1]{\left|#1\right|} % Defination of absolute value

\geometry{
  body={6.5in, 9.0in},
  left=1.0in,
  top=1.0in
}

\tikzstyle{decision} = [diamond, draw, fill=blue!20, 
    text width=4.5em, text badly centered, node distance=3cm, inner sep=0pt]
\tikzstyle{block} = [rectangle, draw, fill=blue!20, 
    text width=6em, text centered, rounded corners, minimum height=4em]
\tikzstyle{line} = [draw, -latex']
\tikzstyle{cloud} = [draw, ellipse,fill=red!20, node distance=3cm,
    minimum height=2em]

%\renewcommand{\baselinestretch}{1.25}
\renewcommand{\baselinestretch}{1}
%\sectionfont{\normalsize}
%\subsectionfont{\normalsize}

\title{User guide for CFD--DEM solver lammpsFoam}

\author[1]{Rui Sun}
\author[2]{Prachant Gupta}
\author[1]{Heng Xiao\thanks{Email: hengxiao@vt.edu}}
\author[2]{Jin Sun}

\affil[1]{Department of Aerospace and Ocean Engineering, Virginia Tech, Blacksburg, Virginia, United
States}

\affil[2]{Institute for Infrastructure and Environment, The University of Edinburgh, UK}

\renewcommand\Authands{ and }

\date{}
\begin{document}

\maketitle

\section{Introduction}

lammpsFoam is an open source CFD--DEM solver.

demFoam is the name of this project combining the LAMMPS and openFoam.

LAMMPS is an open source molecular dynamics code that perform DEM simulations.

OpenFOAM is an open source toolbox for CFD simulations.

The structure of the demFoam is like this:

\begin{lstlisting}
demFoam
|----Allwclean.sh
|----Allwmake.sh
|----cases/
|----documentation/
|----interfaceToLammps/
|----lammpsFoam/
|----readme/
\end{lstlisting}

Allwclean.sh are Allwmake.sh written to help with the compilation of lammpsFoam, the CFD--DEM solver
in demFoam project. Allwmake.sh will help you install lammpsFoam while Allclean.sh will uninstall
lammpsFoam. Note that the source file of LAMMPS is not included in demFoam. We recommend you use
lammps-1Feb14, a version released on February 2014.

All numerical cases are placed in cases/, with verification cases and validations cases.

Some documents are in documentation/, with a basic user manual (userManual.tex).

interfaceToLammps/ and lammpsFoam/ are folders containing the source code of lammpsFoam. To compile
lammpsFoam, the files of interfaceToLammps/ should be copied to LAMMPS-(some version)/src.

\section{How to install lammpsFoam on your linux machine?}

Although Allwmake.sh can automatically help you with the installation of lammpsFoam, it is necessary
to demonstrate individual steps in the compilation.

\subsection{Environment for lammpsFoam}
The CFD-DEM solver lammpsFoam would be successfully installed on a machine with OpenFOAM if you have
all the files and supplemental toolboxes of lammpsFoam (e.g., gcc, MPI). Here are several steps to
do to compile it on your computer:
\begin{enumerate}
    \item Install OpenFOAM on your machine, the up-to-date lammpsFoam is supported for OpenFOAM
    2.1.1.
    \item Copy all files at ``InterfaceToLammps'' to ``lammps-1Feb14/src", or create a soft link of the
    files.
    \item Change to ``lammps-1Feb14/src/STUBS''. Use command ``make''.
    \item Change to the ``lammps-1Feb14/src'' folder. Use command ``make yes-granular'' to install the
    granular package of LAMMPS. Add ``kspace'', ``manybody'', ``molecule'', ``fld'' as well.
    \item Go back to the ``lammps-1Feb14/src'' folder. Use command ``make shanghailinux'' to generate an
    executable file named ``lmp\_shanghailinux'' in this folder.
    \item Stay at ``lammps-1Feb14/src'' folder. Use command ``make shlib'' to generate a new
    ``Makefile.shlib'' file.
    \item Stay at ``lammps-1Feb14/src'' folder. Use command ``make -f Makefile.shlib shanghailinux'' to compile
    LAMMPS to a library. After this step, you will see a file named ``liblammps\_shanghailinux.a".
    \item Change to ``LammpsFoam'' folder. Use command ``wmake libso dragModels'' to compile the drag
    models.
    \item Stay at the same directory. Use command ``wmake'' to compile lammpsFoam.
\end{enumerate}

To check if your lammpsFoam is succesfully compiled, you can:
\begin{enumerate}
    \item Change to the ``cases/auto-testing/test-cases/xiaocase3/'' folder. 
    \item Execute ``./Allrun.sh'' in the command line to see the result.
\end{enumerate}

\section{Supporting files}

There are lot of publications using
lammpsFoam~\cite{sun09,xiao-cicp,gupta11a,gupta11b,gupta11b,gxs11a,gupta13a,gupta12,part1,part2}:

\bibliographystyle{unsrt}
\bibliography{userGuide}

\end{document}

